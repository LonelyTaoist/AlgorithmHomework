\documentclass[UTF8]{ctexart}

\usepackage{amsmath}
\usepackage{amssymb}
\usepackage{algorithm}
\usepackage{algorithmic}

\renewcommand{\thealgorithm}{}

\title{算法分析与设计第三次作业}
\author{57119134 黄浩}
\date{\today}
\begin{document}
\maketitle

\section{题目}

n 个铁管具有重量,被按序存放在 w[i],$1 \le i \le n$。这些铁管将根据它们的顺序,被焊接成一个大的铁管,每个时间任意两个相邻的铁管可以被选中来进行焊接。焊接的代价是焊接的铁管中的重量较大的铁管的重量。例如:w[1]=5,w[2]=1,w[3]=2,如果 1 和 2 先进行焊接,则焊接的代价为 5,然后 3被焊接代价为 6,那么总的焊接代价为 $5+6=11$。但是如果先焊接 2 和 3,再焊 1,那么总的代价为$2+5=7$。
 
1)设计一个动态规划算法去发现最优的焊接顺序,使得整个代价最小和确立迭代关系。

2)将设计的算法应用到一个具体的实例中,去发现最优焊接顺序和其相应的焊接总代价,该实例具有 5 个钢管,w[1]=6,w[2]=2,w[3]=7,w[4]=5,w[5]=8。请给出详细的解决过程。

\noindent 解:

1)

我设计了bestWelding动态规划算法对题目进行求解。算法对重量进行了预先处理,并按照焊接次数从小到大进行迭代计算(即按照焊接1次、2次、3次等等的顺序进行计算)。对于从i到i + j段的最小开销计算过程为,遍历k从0到j - 1,分别求i到i + k段焊接i + k + 1到 i + j 段的代价,并取其中的最小值即为i到i + j段焊接的最小开销,与此同时记录下i到i + j段焊接的位置用于最终复原焊接过程。

在此说明各个变量的意义以及一些简单的初始化。weight[i][j]表示从w[i]到w[j]的总重量,在计算之前weight元素全部初始化为0;dp[i][j]表示从w[i]焊接到w[j]的最小代价,在计算之前dp元素初始化为INT\_MAX(但是dp[m][m]元素初始化为0,即dp对角线上的元素初始化为0);welding[i][j]表示从i到j段的焊接位置;所以最终dp[1][n]就是我们所求的从w[1]到w[n]的焊接最小代价,并且我们可以通过回溯welding层层寻找我们的焊接过程。

伪代码如下:

\begin{algorithm}[h]
	\caption{bestWelding}
	\begin{algorithmic}[1]
	\STATE //计算各个区间的重量
	\FOR {i = 1 to n}
		\FOR {j = i to n}
			\STATE $weight[i][j] = weight[i][j - 1] + w[j]$ //weight初始化为0
		\ENDFOR
	\ENDFOR
	\STATE //动态规划计算
	\STATE //j为区段差值,i为区段开始值,k为区段分割的位置
	\FOR {j = 1 to n - 1}
		\FOR {i = 1 to n - j}
			\FOR {k = 0 to j - 1}
				\STATE $w = max(weight[i][i + k], weight[i + k + 1][i + j])$ //两端更重值
				\STATE $cost = dp[i][i + k] + dp[i + k + 1][i + j] + w$//焊接开销
				\IF {$dp[i][i + j] > cost$}
					\STATE $dp[i][i + j] = cost$
					\STATE $welding[i][i + j] = i + k$//记录焊接位置
				\ENDIF
			\ENDFOR
		\ENDFOR
	\ENDFOR
	\STATE return dp[1][n]
	\end{algorithmic}
\end{algorithm}

2)

模拟算法过程如下:(省去一些预处理计算过程)

焊接1次:

$dp[1][2] = 6;welding[1][2] = 1$

$dp[2][3] = 7;welding[2][3] = 2$

$dp[3][4] = 7;welding[3][4] = 3$

$dp[4][5] = 8;welding[4][5] = 4$

焊接2次:

$dp[1][3] = min(14,16) = 14;welding[1][3] = 2$

$dp[2][4] = min(16,19) = 16;welding[2][4] = 3$

$dp[3][5] = min(19, 21) = 19;welding[3][5] = 4$

焊接3次:

$dp[1][4] = min(29,25,30) = 25;welding[1][4] = 2$

$dp[2][5] = min(30,28,39) = 29;welding[2][5] = 3$

焊接4次:

$dp[1][5] = min(45,37,45,51) = 37;welding[1][5] = 3$

所以我们回溯welding:

$welding[1][5] = 3$

\quad $welding[1][3] = 2$

\qquad $welding[1][2] = 1;welding[2][3] = 2$

\quad $welding[4][5] = 4$

综上所述,我们焊接此w[1]到w[5]的最小开销为37。

我们回溯的焊接过程为:

1->2; 2->3; 4->5; 3->4

\end{document}